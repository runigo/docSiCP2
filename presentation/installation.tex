\section{Installation de SiCP2}
%
\subsection{Exécutables pour Windows}
%
\subsubsection{Version portable}
%
\begin{itemize}[leftmargin=1cm, label=\ding{32}, itemsep=0pt]
%
\item Télécharger la version portable sur le site https://cphysique.github.io/
\item Vérifier le fichier
	\begin{itemize}[leftmargin=1cm, label=\ding{32}, itemsep=0pt]
	\item Ouvrir une invite de commande (\texttt{cmd})
	\item Se déplacer dans le dossier de téléchargement (\texttt{cd Downloads})
	\item Calculer la somme de contrôle (\texttt{certutil -hashfile SiCP2.4.4.zip SHA256})
	\item Vérifier la correspondance de la somme calculée avec la somme donnée sur le site.
	\end{itemize}
\item Décompresser le fichier (par exemple avec 7-zip https://www.7-zip.org/)
\item Le répertoire obtenu contient le fichier exécutable et les fichiers nécessaire à l'exécution.
\item Il est alors possible de déplacer ce répertoire, de créer un lien vers l'exécutable et de changer l'icône du lien en utilisant l'image SiCP2.ico.
\end{itemize}
%
%\begin{center}
%\includegraphics[width=.9\textwidth]{./illustration/SiCP2}
%\end{center}
%
\subsubsection{Installation automatique}
%
\begin{itemize}[leftmargin=1cm, label=\ding{32}, itemsep=0pt]
%
\item Télécharger l'installateur sur le site https://cphysique.github.io/
\item Vérifier le fichier
	\begin{itemize}[leftmargin=1cm, label=\ding{32}, itemsep=0pt]
	\item Ouvrir une invite de commande (\texttt{cmd})
	\item Se déplacer dans le dossier de téléchargement (\texttt{cd Downloads})
	\item Calculer la somme de contrôle (\texttt{certutil -hashfile setupSiCP2.4.3.exe SHA256})
	\item Vérifier la correspondance de la somme calculée avec la somme donnée sur le site.
	\end{itemize}
\item Exécuter l'installateur, cocher la case \texttt{ajouter un lien sur le bureau}
\item Il est possible de changer l'icône du lien en utilisant l'image SiCP2.ico contenu dans le répertoire d'installation (\texttt{C/Program Files/SiCP2})
%
\end{itemize}
%
%
%
\subsection{Compilation du code source}
%
Le code source se trouve sur \texttt{https://github.com/Cphysique/SiCP2}
%
\begin{center}
\includegraphics[width=.6\textwidth]{./presentation/CphysiqueSiCP2}
\end{center}
%
Le bouton vert de la page sur https://github.com/Cphysique/SiCP2/ permet de télécharger le code source.
%
\begin{center}
\includegraphics[width=.3\textwidth]{./presentation/CphysiqueSiCP2code}
\end{center}
%
Cette section traite de l'installation des simulateurs SiTS2, SiCF2 et SiCP2 sur un système d'exploitation de type debian. Le téléchargement se fait avec un navigateur internet, la compilation et l'exécution se font dans un terminal. L'installation des outils de compilation nécessite les privilèges du super-utilisateur.
\begin{itemize}[leftmargin=1cm, label=\ding{32}, itemsep=0pt]
\item {\bf Installation des outils de compilation}
Avec les droits du super-utilisateur
	\begin{itemize}[leftmargin=1cm, label=\ding{32}, itemsep=0pt]
	\item \texttt{apt-get install gcc make libsdl-dev}
	\item Pour les versions 2 des simulateurs, installer la librairie SDL2 :
	\item \texttt{apt-get install libsdl2-dev}
	\end{itemize}
\item {\bf Téléchargement des sources}
	\begin{itemize}[leftmargin=1cm, label=\ding{32}, itemsep=0pt]
	\item Télécharger les fichiers \texttt{.zip} sur github
		\begin{itemize}[leftmargin=1cm, label=\ding{32}, itemsep=0pt]
		\item \texttt{https://github.com/runigo/SiCP2/archive/master.zip}
		\item \texttt{https://github.com/runigo/SiCF2/archive/master.zip}
		\item \texttt{https://github.com/runigo/SiTS2/archive/master.zip}
		\end{itemize}
	\item Décompresser les fichiers \texttt{.zip}
		\begin{itemize}[leftmargin=1cm, label=\ding{32}, itemsep=0pt]
		\item \texttt{unzip SiCP2-master.zip}
		\item \texttt{unzip SiCF2-master.zip}
		\item \texttt{unzip SiTS2-master.zip}
		\end{itemize}
	\end{itemize}
\item {\bf Compilation}
	\begin{itemize}[leftmargin=1cm, label=\ding{32}, itemsep=0pt]
	\item La commande \texttt{make} dans le répertoire des sources produit un fichier exécutable :
		\begin{itemize}[leftmargin=1cm, label=\ding{32}, itemsep=0pt]
		\item \texttt{SiCP2} pour SiCP
		\item \texttt{SiCF2} pour SiCF
		\item \texttt{SiTS2} pour SiTS
		\end{itemize}
	\end{itemize}
%
\item {\bf Exécution}
	\begin{itemize}[leftmargin=1cm, label=\ding{32}, itemsep=0pt]
	\item En ligne de commande, avec d'éventuelles options
		\begin{itemize}[leftmargin=1cm, label=\ding{32}, itemsep=0pt]
		\item \texttt{./SiCP2 [OPTION]}
		\item \texttt{./SiCF2 [OPTION]}
		\item \texttt{./SiTS2 [OPTION]}
		\end{itemize}
	\item La fenêtre graphique donne une représentation de la simulation,
	\item Le terminal affiche les informations.
	\end{itemize}
\end{itemize}

%%%%%%%%%%%%%%%%%%%%%%%%%%%%%%%%%%%%%%%%%%%%%%%%%%%%%%%%%%%%%%%%%%%%%%%%%%%%%%%%%%%%%%%%%%%%%

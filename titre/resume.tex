\begin{center}
\Large
Résumé
\normalsize
\end{center}
\vspace{3cm}
\begin{itemize}[leftmargin=1cm, label=\ding{32}, itemsep=21pt]
\item {\bf Objet : }Ce document accompagne le programme SiCP2.
\item {\bf Contenu : }Il contient un manuel d'installation et d'utilisation ainsi qu'un certain nombre de développements théoriques.
\item {\bf Public concerné : }Les enseignants, les étudiants et les passionnés de physique et d'informatique.
\end{itemize}

\vspace{3cm}

SiCP2 est un simulateur numérique d'équations physiques offrant une représentation graphique et une interaction dynamique avec les paramètres physiques. Destinés à un usage ludique et pédagogique, il permet de visualiser le comportement des systèmes physiques simulés. Cette documentation accompagne ce programme.

\begin{itemize}[leftmargin=1cm, label=\ding{32}, itemsep=11pt]
\item Les deux premiers chapitres présentent SiCP2, présente les procédures d'installation et précisent les commandes permettant l'interaction avec le programme.
\item Les deux chapitres suivants fournissent un certain nombre de développements théoriques liés au phénomènes physiques et à la numérisation des équations.
\item Enfin, le dernier chapitre rassemble les informations liées à la structure du code source.
\end{itemize}

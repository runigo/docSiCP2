%
%%%%%%%%%%%%%%%%%%%%%
\section{Valeurs implicites}
%%%%%%%%%%%%%%%%%%%%%
%
%
\subsection{Réglage de dt, durée et pause}
Une incrémentation du système correspond à une avancée dans le temps de \texttt{dt}. La longueur des pendules est fixé à 25 cm afin de battre la seconde :
Période égale à une seconde
\begin{center}
	\begin{tabular}{rcccl}
	T & = & 2$\pi \sqrt{l/g}$ & = & 1\\
	g/l & = & 4$\pi^2$ & = & 39,478\\
	l & = & 0,25 cm &\\
	\end{tabular}
\end{center}
Empiriquement, un affichage graphique par 30 ms, est obtenue avec une pause de l'ordre de 25ms. Si à chaque affichage correspond à une centaine d'incrémentation de dt, 
\begin{center}
	\begin{tabular}{rcl}
	dt $\times$ duree & = & delay\\
	dt $\times$ 100 & = & 0,03\\
	dt & = & 0,0003\\
	\end{tabular}
\end{center}
La valeur de\texttt {duree} peut être changée dynamiquement avec les touches \texttt{F11} et \texttt{F12} afin de faire varier la vitese de la simulation. La valeur de dt peut être réglée avec l'option \texttt{dt} au démarrage du programme, celle de delay par l'option \texttt{pause}.
Dans SiCP, les valeurs implicites de \texttt{dt} et \texttt{duree} sont égale à 0,0003 et 91, celle de pause est égale à 25. Ces valeurs peuvent être affinée suivant le microprcesseur afin d'avoir un pendule qui bat la seconde.
%
%
\subsection{Limite infinie}
La touche \texttt{v} supprime les frottements sauf pour les derniers pour lesquels les frottements s'accroissent. Ceci permet d'obtenir une extrémité "absorbante".
\begin{itemize}[label=\ding{32}, leftmargin=2cm]
\item pendule de  « précédent »  à  «  nombre $\times$ 5 / 6 » : dissipation de 10 à 1,
\item pendule précédents : dissipation = 0,0
\end{itemize}
%
%
\subsection{dt et dissipation maximale}
La touche \texttt{v} supprime les frottements sauf pour les derniers pour lesquels les frottements s'accroissent. Ceci permet d'obtenir une extrémité "absorbante".
\begin{center}
	\begin{tabular}{rcl}
	dt $\times$ DISSIPATION\_MAX & = & constante\\
	0.0003 $\times$ 333 & = & 0,0999\\
	dissipation maximale & = & 0,0999/dt\\
	\end{tabular}
\end{center}
%
%
\subsection{Limitation des valeurs des variables}
%
\subsubsection{Paramètres physiques}
%
Au delà de certaines valeurs de certain paramètres dynamiques, la simulation s'éloigne du comportement physique.
%
\begin{itemize}[label=\ding{32}, leftmargin=2cm]
\item Pour des raisons de discrétisation
\item En raisons de possibles erreurs d'algorithme
\item En raisons de possibles erreurs d'écriture
\end{itemize}
%
Aussi, le fichier \texttt{donnees/constantes.c} contient des valeurs maximale et minimale de certains paramètres. Ces bornes permettent de consolider le comportement des programmes.

En particulier, dans le calcul de la représentation du graphe de SiCP, les coordonées polaires du point de vue sont bornées, $\phi$ ne peut pas être égale à zéro, sa valeur minimale est égale à \texttt{EPSILON} afin d'éviter un plantage due à l'algorithme simplifiée de la représentation graphique.
%
\subsubsection{Paramètres dynamiques}
%
\begin{itemize}[label=\ding{32}, leftmargin=2cm]
\item Vitesse des mobiles
\item Distance entre les mobiles
\item Énergie
\end{itemize}
%
Actuellement, l'angle des pendules de SiCP est borné grâce à une fonction de jauge ramenant la position du premier pendule entre -$\pi$ et $\pi$.
Un test sur la valeur de l'énergie et la limitation de celle-ci permetrait sans doute de consolider davantage le programme. En effet, parmi les bug connus, une valeur trop grande des variables apparait en premier lieu dans la valeur de l'énergie totale.

Ainsi, afin de limiter la vitesse des mobiles, la position des mobiles pourrait être divisées par deux dans les équations linéaires lorsque l'énergie totale est supérieur à \texttt{ENERGIE\_SECURITE}.
%\subsubsection{Énergie maximale}
%La valeur de l'énergie permetrait de limiter un certain nombres d'erreurs.
%
%Lorsque l'énergie est supérieur à ENERGIE\_SECURITE, La position des mobiles pourrait être divisées par deux dans les équations linéaires
%\begin{center}
%	\begin{tabular}{rcl}
% &  &1 009 200 790 976 791 136 174 080\\
% &  &  585 891 747 701 731 153 149 952\\
%      999000555444333222111\\
%\multicolumn{3}{c}{\#define ENERGIE\_SECURITE 777666555444333222111}\\
%	\end{tabular}
%\end{center}
%
